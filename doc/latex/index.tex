\hypertarget{index_section_tdm}{}\subsection{Table des matières}\label{index_section_tdm}

\begin{DoxyItemize}
\item \hyperlink{page_README}{R\+E\+A\+D\+ME}
\item \hyperlink{page_changelog}{Changelog}
\item todo
\item \hyperlink{page_about}{A propos}
\item \hyperlink{page_licence}{Licence G\+PL}
\end{DoxyItemize}

\subsubsection*{Programme Qt}

Le programme réalisé avec Qt 5.\+11.\+2 permet de communiquer avec une sonde équipée de différents capteurs.

L\textquotesingle{}I\+HM affiche les informations des capteurs dans l\textquotesingle{}onglet Données. Il est possible de consulter les relevés de mesures sous forme de graphiques.



Il est possible de consulter les données météos d\textquotesingle{}une ville dont il est possible de saisir le nom. Un affichage des coordonnées G\+PS est dipsonible.



Tous les échanges de trame s\textquotesingle{}affichent dans l\textquotesingle{}onglet opérateur où il est également possible de piloter la led manuellement selon le protocole.



La communication se fera au choix de l\textquotesingle{}utilisateur, soit par liaison série soit par communication Bluetooth. La communication via Wi\+Fi n\textquotesingle{}est pas implémenté dans ce programme.



\subsubsection*{Sonde E\+S\+P32-\/\+Weather}

La carte E\+S\+P32-\/\+Weather est une sonde construite autour d\textquotesingle{}un E\+S\+P32 et équipée d\textquotesingle{}un module {\bfseries Blue\+Dot} I2C, qui intègre un capteur d\textquotesingle{}éclairement lumineux {\bfseries T\+SL 2591} et un capteur {\bfseries B\+M\+E280} (température, humidité et pression atmosphétrique), et d\textquotesingle{}une Led Bicolore. Les mesures sont affichées périodiquement sur l\textquotesingle{}écran {\bfseries O\+L\+ED} de la carte.

La sonde communique aussi via le Wi\+Fi, le Bluetooth et la liaison série. Le même protocole est utilisé pour les trois modes de communication. L\textquotesingle{}écran de la sonde affiche l\textquotesingle{}adresse IP et le numéro de port utilisés pour une communication Wi\+Fi et l\textquotesingle{}adresse M\+AC de l\textquotesingle{}interface Bluetooth.

La carte a été réalisée par des étudiants d\textquotesingle{}EC et le programme de l\textquotesingle{}E\+S\+P32 par un professeur.



\subsubsection*{Auteurs}

{\itshape Fabien} Bounoir (IR) \href{mailto:bounoirfabien@gmail.com}{\tt bounoirfabien@gmail.\+com} 
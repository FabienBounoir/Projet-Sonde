\subsubsection*{Nom \+: Mini-\/projet Qt Sonde E\+S\+P32}

\subsubsection*{Numéro de version \+: 4.\+0}

\subsubsection*{Auteurs}

{\itshape Fabien} Bounoir (IR) \href{mailto:bounoirfabien@gmail.com}{\tt bounoirfabien@gmail.\+com}

\subsubsection*{Description}

Le programme a été réalisé avec Qt 5.\+11.\+2.

Fichier {\ttfamily .pro} \+:


\begin{DoxyCode}
00001 QT       += core gui \(\backslash\)
00002             serialport network \(\backslash\)
00003             bluetooth \(\backslash\)
00004             positioning \(\backslash\)
00005             charts
00006 
00007 greaterThan(QT\_MAJOR\_VERSION, 4): QT += widgets
00008 
00009 TARGET = Sonde
00010 TEMPLATE = app
00011 
00012 CONFIG += c++11
00013 
00014 SOURCES += \(\backslash\)
00015         main.cpp \(\backslash\)
00016         ihm.cpp \(\backslash\)
00017     transmission.cpp \(\backslash\)
00018     esp32.cpp \(\backslash\)
00019     meteo.cpp \(\backslash\)
00020     graphique.cpp \(\backslash\)
00021     gps.cpp
00022 
00023 HEADERS += \(\backslash\)
00024         ihm.h \(\backslash\)
00025     transmission.h \(\backslash\)
00026     esp32.h \(\backslash\)
00027     meteo.h \(\backslash\)
00028     graphique.h \(\backslash\)
00029     gps.h
00030 
00031 FORMS += \(\backslash\)
00032         ihm.ui
00033 
00034 CONFIG(release, debug|release):DEFINES+=QT\_NO\_DEBUG\_OUTPUT
\end{DoxyCode}


\subsubsection*{Protocole}


\begin{DoxyCode}
00001 
      SONDE;TEMPERATURE;UNITE;RESSENTIE;UNITE;HUMIDITE;UNITE;ECLAIREMENT;UNITE;PRESSION;UNITE;ALTITUDE;UNITE;\(\backslash\)n
00002 LED;ETAT LED ROUGE;ETAT LED VERTE;ETAT;COULEUR;\(\backslash\)n
\end{DoxyCode}


Exemple \+:


\begin{DoxyCode}
00001 SONDE;20.8;C;20.0;C;41;%;997;lux;1007;hPa;52;m;\(\backslash\)n -> Température 20,8 °C, Ressentie 20 °C, Humidité 41
       %, un éclairement de 997 lux, une pression atmosphérique de 1007 hPa et d'une altitude évaluée à 52 m
00002 LED;1;0;1;1;\(\backslash\)n -> Le Led est allumée en rouge
\end{DoxyCode}


{\itshape Remarques \+:}


\begin{DoxyItemize}
\item Les valeurs de température sont précisées au dixième de degré.
\item Un booléen est égal à 0 pour false et 1 pour true.
\item Les codes de couleur pour la Led sont \+:
\begin{DoxyItemize}
\item Aucune (éteinte) = 0
\item Rouge = 1
\item Verte = 2
\item Orange = 3
\end{DoxyItemize}
\end{DoxyItemize}

Les clients connectés ont la possibilité d\textquotesingle{}envoyer une requête pour commander la led \+:


\begin{DoxyCode}
00001 SET LED commande\(\backslash\)n
\end{DoxyCode}


Le champ commande peut prendre les valeurs suivantes \+:


\begin{DoxyCode}
00001 SET LED ON\(\backslash\)n -> allume la Led dans sa couleur courante
00002 SET LED OFF\(\backslash\)n -> éteint la Led
00003 SET LED 0\(\backslash\)n -> éteint la Led
00004 SET LED 1\(\backslash\)n -> allume la Led en rouge
00005 SET LED 2\(\backslash\)n -> allume la Led en vert
00006 SET LED 3\(\backslash\) -> allume la Led en orange
00007 SET LED ROUGE\(\backslash\)n -> allume la Led en rouge
00008 SET LED VERT\(\backslash\)n -> allume la Led en vert
00009 SET LED VERTE\(\backslash\)n -> allume la Led en vert
00010 SET LED ORANGE\(\backslash\)n -> allume la Led en orange
\end{DoxyCode}


{\itshape Remarque \+: la requête est insensible à la casse.} 